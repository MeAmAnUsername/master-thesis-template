% !TEX root = document.tex

\section{Introduction}
\label{sec:introduction}

Transforming some inputs into some outputs is often not done with a single program, but with multiple programs that have to executed in sequence, passing the result between them.
This is called a pipeline, and for the purposes of this thesis, the steps in such a pipeline are called tasks.
For smaller projects, such a pipeline can be expressed in a command line script like a bash script or a bat script, but as the project grows such build scripts often become unmaintainable.
Additionally, executing the full pipeline will take a long time, even if nothing changed.
Build systems aim to keep build scripts maintainable by providing a concise way to express dependencies of and between tasks.
They also keep track of which inputs changed in order to build incrementally, which saves a lot of time in the case of small changes.

\todo{explain what precise dynamic dependencies are (and why we care?)}
A build system is precise if it keeps track of the exact dependencies of a task.
A dynamic dependency is a dependency that can only be resolved at runtime.
Almost no build system supports precise dynamic dependencies in a concise way.\todo{verify this claim, back it up with some references, or make it more specific so that it is true}
PIE does support precise dynamic dependencies.
PIE is implemented as a framework in Java.
As such, it has a lot of boilerplate.
Java is also poor for expressing concepts from the domain of pipelines.
To solve these issues, the PIE DSL was developed.
The PIE DSL allows a pipeline developer to express the relevant parts of a task definition in a language that allows easily expressing common concepts in pipelines.
Because the DSL has significantly less boilerplate, it saves the developer time, is easier to read and has less duplication which improves maintainability and prevents bugs. 
The concise PIE DSL task definition is compiled to Java, which generates all the Java boilerplate.

While the PIE DSL and the compiler had implementations, they had several shortcomings:
\begin{enumerate}
  \item Existing compilers were not maintainable.
  \item The DSL does not scale to larger projects, e.g. projects that consist of multiple language projects.
  \item The DSL is implemented in NaBL2, which has a few limitations in expressiveness.
  \item The DSL has limitations in expressiveness
  \begin{enumerate}
    \item No way to express core concepts of PIE, e.g. Suppliers and Results.
    \item No way to declare injected values for tasks.
  \end{enumerate}
\end{enumerate}

We set out to improve the PIE DSL with the following objectives:
\begin{itemize}
  \item Solve these problems
  \item Keep the PIE DSL general
  \begin{itemize}
    \item It could also compile to another language if another compiler and PIE framework were written in that language.
    \item It should work for pipelines in general, not just Spoofax pipelines.
  \end{itemize}
  \item Since PIE is meant to give realtime feedback in Spoofax, it should be reasonably performant
  \item Keep the PIE DSL extendable. Do not add features that are incompatible with future extensions of the language where this is reasonably possible.
\end{itemize}

In the end, this thesis [has]  the following contributions:
\begin{itemize}
  \item a new compiler for the DSL
  \item implemented static semantics of the DSL in Statix
  \item Add modules, context parameters (and generics?) to the DSL
  \item Evaluate the new DSL in three case studies:
  \begin{itemize}
    \item simple transformation from Tiger
    \item database pipelines at Oracle
    \item testing pipelines at Oracle
  \end{itemize}
  \item Evaluation of performance for generated code (?)
\end{itemize}

The rest of this thesis is set up as follows.
Section 2 gives an explanation of our use cases and explains the problems in more detail.
Section 3 lists the improvements that were made to the PIE DSL.
Section 4 evaluates the new DSL with three case studies.
It also compares expressing a pipeline in the DSL or directly in the Java framework.
Section 5 lists related work. \todo{slightly more descriptive: "puts this work into context by comparing it to solutions for similar problems"}
Section 6 concludes this thesis.
