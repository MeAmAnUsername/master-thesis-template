% !TEX root = document.tex

\chapter{\label{chap:introduction}Introduction}

todo{short introduction: what is spoofax?}
todo{short introduction: what is PIE?}

Using the PIE framework requires a lot of boilerplate.
To mediate this, the PIE DSL is a DSL to specify PIE pipelines.
It compiles to Java.
The PIE DSL was created by Gabriël Konat in todo{reference} as part of his work on the PIE framework.
This included a compiler from PIE DSL to Kotlin.
For various reasons compiling to Kotlin was deemed undesirable, so Ivo Wilms implemented a compiler from the DSL to Java todo{is there a reference? If not: add footnote to explain why}.

\section{\label{sec:introduction__problem_statement}Problem statement}
The PIE DSL was not [powerful|expressive|extensive] enough to be used in many real world use cases.
The Java compiler implemented by used string interpolation, which (as it turned out) is "suboptimal" \footnote{it's severely limiting, don't do it}.

Problems:
\begin{enumerate}
  \item Existing compilers are [bad]
  \begin{itemize}
     \item Compiler to Kotlin is outdated, compiling to Kotlin is undesirable.
     \item Compiler to Java compiles via string interpolation, which is very limiting.
   \end{itemize}
  \item Single file only (does not scale to multi-language project setups, see [ref: case study database pipelines])
  \item Expressiveness
  \begin{enumerate}
    \item No way to express core concepts of PIE, e.g. Suppliers [also maybe Results]
    \item No way to declare injected values for tasks
  \end{enumerate}
\end{enumerate}

Objectives
\begin{itemize}
  \item Solve these problems
  \item Keep the PIE DSL general
  \begin{itemize}
    \item It could also compile to another language if another compiler and PIE framework were written in that language.
    \item It should work for pipelines in general, not just Spoofax pipelines.
  \end{itemize}
\end{itemize}

\section{\label{sec:introduction__solution}Solution}
\begin{enumerate}
  \item Implement a compiler that compiles to Java using Java ASTs
  \item Implement a module system
  \item Extend the old PIE DSL with new constructs
  \begin{enumerate}
    \item Add `supplier<T>` construct
    \item Add context parameters construct
  \end{enumerate}
\end{enumerate}
