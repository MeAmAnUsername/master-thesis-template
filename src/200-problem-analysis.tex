% !TEX root = document.tex

\section{Problem analysis}
\label{sec:problem_analysis}

\subsection{Background information}
\label{subsec:problem_analysis__background}

\subsubsection{ATerms}
\label{subsubsec:problem_analysis__background__aterms}

Programs are expressed in human readable syntax, but this is hard and inefficient to manipulate directly.
Instead, code is parsed to an \ac{AST}, which throws away the syntax and keeps the parts of a program that can change.
In Spoofax, these \acp{AST} are represented using ATerms.
An ATerm consists of a constructor, arguments and possibly some annotations.
As an example, the expression \inlinecode{x * 9 - 3} is represented as \inlinecode{Minus(Times(Var("x"), Int("9")), Int("3"))}.
Spoofax meta-languages can pattern-match on these constructors to define rules, such as rules in Stratego to transform a constructor or rules in Statix to define the static semantics.

\subsubsection{Ways to compile}
\label{subsubsec:problem_analysis__background__ways_to_compile}

There are two ways to compile from a source language to a target language.

The simplest way to compile a source language to a target language is to compile fragments of the source language into fragments of the target language, write a template in the target language and insert the compiled fragments into that template.
This method is called string interpolation.
It has the advantage of being fairly easy to read and write, but it has several disadvantages.
A major disadvantage of string compilation is that it does not have static checks.
This means that the Stratego compiler \todo{general term for "compiler compiler"} does not check that the generated code will form a correct program, or even that it is syntactically correct.
A simple typo in the generated code is only caught at runtime instead of compile time.
This leads to longer development times and transformation rules that are tested on a small subset of possible inputs, instead of verified for all possible inputs.
String interpolation is also inefficient in case one wants to do some post-processing on the generated code, such as performing optimizations.
To perform such optimizations, the code needs to be parsed to an AST, so concatenating strings only to then parse the strings is inefficient.

To avoid bugs due to typos in the generated code fragments, one can transform to an \ac{AST} of the target language instead.
The \ac{AST} is then transformed to code by a generated prettyprinter.
Additionally, it is easier to check correctness of an \ac{AST}.
Stratego currently verifies the arity of constructors.
Verifying the arguments to a constructor are of the right type (e.g. checking that \inlinecode{x} and \inlinecode{y} in \inlinecode{Add(x, y)} are expressions instead of, say, import statements) is currently a work in progress.
Transforming to an \ac{AST} also means that the transformed program is immediately ready for post-processing, no parsing required.

\subsubsection{Language projects in Spoofax}
\label{subsubsec:problem_analysis__background__spoofax_language_projects}

Language engineering is the discipline of specifying and implementing programming languages.
The state of the art right now is to implement languages ad-hoc, i.e.\ most major languages implement most of their parsing, syntax highlighting, static analysis, etc. from scratch.\todo{add reference}
A language workbench is an \ac{IDE} for developing programming languages.
Spoofax is a language workbench with the goal of abstracting over the implementation details of a language, and allowing language developers to focus solely on the design of their language, instead of how it should be implemented.
It does this using multiple declarative meta-\acp{DSL}.
The language developer can write a specification of their language in these \acp{DSL} and Spoofax automatically builds an implementation from these specifications.

A language in Spoofax is represented as a language project.
Language projects contain several types of files.
\Ac{SDF3} files specify the grammar of the language.
The productions in \ac{SDF3} files are also used to generate constructors used in the ATerms of the program \acp{AST}.
Static semantics\footnote{Static semantics of a language refer to basic stuff like \inlinecode{"Hello"} is a string, adding two strings together yields another string, if you call a function that takes a single string then you should give it a single argument and that argument has to be a string, etc.}
can be implemented using either \ac{NaBL2} or Statix.
Both are meta-\acp{DSL} for specifying the static semantics, Statix is the successor of \ac{NaBL2}.
Languages can also have Stratego files.
Stratego is a language for executing transformations on programs.
Examples of transformations are optimizing a program, reducing it to a normal form and compiling to another language.
Language projects include \ac{ESV} files that specify editor configuration settings like syntax highlighting (which is also autogenerated, but this allows a language developer to customize it) and editor actions.

Finally, a \filename{metaborg.yaml} file denotes a directory as a language project.
This file specifies project configurations like what \ac{SDF3} compiler to use.
It also allows exporting parts of this language project and importing exported parts of other language projects.
This makes language composition possible.
For example, the \ac{PIE} language project imports the Java grammar to compile the \ac{PIE} \ac{DSL} to Java.

\subsubsection{Compilation with Stratego}
\label{subsubsec:problem_analysis__background__stratego}

Stratego is a meta-language to express program transformations.
Examples are optimizations (\inlinecode{5 + x + 10 * x + 8} to \inlinecode{18 + 11 * x}) and compiling one language to another: \inlinecode{for (int i = 0; i<10; i++) \{...\}} to \inlinecode{for i in 0..9: ...}.
Stratego uses rules to define transformations from one constructor to another.
For example, the following rule combines two integer literals into a single value: \inlinecode{add-literal-ints: Plus(Int(x), Int(y)) -> Int(x + y)}.
There can be multiple rules with the same name.
Stratego will try them until one of them succeeds or all failed.
This is like an implicit pattern match against constructors:
\begin{lstlisting}
  fold-constants: And(True(), e2) -> e2
  fold-constants: And(False(), _) -> False()
  fold-constants: And(e1, True()) -> e1
  // No And(e1, False()) -> False() because that changes semantics, it won't execute e1 anymore
  fold-constants: If(True(), body_true, _) -> body_true
  fold-constants: If(False(), _, body_false) -> body_false
\end{lstlisting}

The \inlinecode{fold-constants} rule above will perform constant folding on booleans and if statements where the condition is a constant.
However, if we apply this rule to the term \inlinecode{If(And(True(), True()), Print("It's true"), Print("It's false"))}, it fails.
The \inlinecode{And} constructors don't match because the top level constructor is \inlinecode{If}.
And the \inlinecode{If} constructors don't match because the condition is neither \inlinecode{True()} nor \inlinecode{False()}, it is \inlinecode{And(...)}.

What we want instead, is to apply this rule to any matching subterm of the \ac{AST}.
To do this, Stratego has strategies.
A strategy defines how to apply a rule to an \ac{AST}.
In this case, we can try to apply it to all subterms: \inlinecode{bottomup(try(fold-constants))}.
This first applies the rule to the subterms to produce \inlinecode{If(True(), Print("It's true"), Print("It's false"))}, and then applies the rule to the term itself to produce the final result \inlinecode{Print("It's true")}.

\subsubsection{Foreign functions in \ac{PIE} \ac{DSL}}
\label{subsubsec:problem_analysis__background__foreign_functions}

The \ac{PIE} \ac{DSL} can interoperate with Java by calling Java methods.
To call a Java method, it needs to be declared in the \ac{PIE} program.
A declaration looks like this:
\begin{lstlisting}
  func sign(x: int) -> int = foreign java java.lang.Math.signum
\end{lstlisting}
The part before the \inlinecode{=} is a normal \ac{PIE} function declaration.
It defines a function \inlinecode{sign} that takes an integer x and returns an integer.
The part after the \inlinecode{=} specifies the function definition.
\inlinecode{foreign java} designates this as a function that is declared in Java, and \inlinecode{java.lang.Math.signum} is the fully qualified name to the class and the function name.
The definition does not need the Java parameters because the \ac{PIE} parameters are mapped to Java parameters automatically.

\subsection{Problems}
\label{subsec:problem_analysis__problems}

\todo{Should I talk about these problems in the present tense or past tense? ("The PIE DSL cannot express ..." vs. "The PIE DSL could not express ...")}
The problems with the \ac{PIE} \ac{DSL} fall into two categories.
There are problems with the language design and problems with the language implementation.

\subsubsection{Lack of scalability}
\label{subsubsec:problem_analysis__problems__scalability}

A \ac{PIE} program used to be written entirely in a single file, referring to other files was not possible.
In Spoofax, we want one \ac{PIE} file per language project, \ac{PIE} is limited to a single language project at a time.
In practice, compilers often need at least two language projects: a language project with the target language and a language project with the source language.
The compiler code itself is then either included in the source language or defined in a separate third language project.
Finally, a project could just consist of multiple projects.
An example is the Green-Marl compiler, which is one of the case studies (see \todo{add reference to Green-Marl pipelines case study}).
In conclusion, we want a principled way to refer to other \ac{PIE} files in order to use \ac{PIE} in bigger projects with multiple languages.

\subsubsection{Limitations in expressiveness}
\label{subsubsec:problem_analysis__problems__expressiveness}

The \ac{PIE} framework is under active development.
This sometimes adds new requirements on things to express using the \ac{PIE} \ac{DSL}.

At the start of the thesis, the \ac{PIE} framework used to throw exceptions to signal pipeline failure.
Checked exceptions in Java hurt composability of tasks, because checked exceptions from dependencies either need to be declared or caught.
Throwing exceptions to signal pipeline failure also has some semantic issues: exceptions are meant to signify an exceptional condition.
Malformed input is not unexpected, so this should not lead to an exception.
The \ac{PIE} framework solved this encoding the status of a task in its return value.
While a task can encode this in any way it likes, the standard way in \ac{PIE} is to use the provided library class Result<T, E> that encodes either success with a normal value T or failure with an exception E.
These results are [composable]: a task can take a result and either do some computation if it has a value, or just pass along the exception if it had an exception.
This allows easy composition of tasks, where one only needs to check the final result for success or failure.
While this undoubtedly improves the \ac{PIE} framework, it presents a problem for the \ac{PIE} \ac{DSL}: it now needs to support Result, and other generic library classes like Supplier<T>, Option<T>, <TODO: more examples?>.
The \ac{PIE} \ac{DSL} does not support generics.
Because foreign Java methods are declared in a \ac{PIE} program with \ac{PIE} \ac{DSL} syntax, it is barely possible to interface with generic classes.
The only thing that can be declared is an instance of the generic parameter. That is not enough for classes that are used with different generic arguments, such as these library classes.
Additionally, it would be good if the \ac{PIE} \ac{DSL} could interoperate with arbitrary generic classes, not just generic classes from the \ac{PIE} library.

The other new requirement from the \ac{DSL} stems from a property of tasks: not everything that is required to execute a task is an argument.
Sometimes values are hardcoded into the task\footnote{In the actual code these values are not hardcoded but added during \inlinecode{TaskDef} construction using dependency injection}.
An example is the StrategoRuntime that is used to execute Stratego strategies.
This runtime is specific to each language and is added to the task as an instance variable.
The \ac{PIE} \ac{DSL} did not have any way to specify such an instance variable.

We would like to extend the language with new syntax and semantics in order to handle these uses.

\subsubsection{NaBL2 limits static semantics}
\label{subsubsec:problem_analysis__problems__nabl2}

As mentioned in the previous problem (\ref{subsubsec:problem_analysis__problems__expressiveness}), the \ac{PIE} \ac{DSL} could not interoperate with generic classes in Java.
The simplest solution is to add support for generic classes to \ac{PIE} \ac{DSL} as well.
However, \ac{PIE} \ac{DSL} is implemented using NaBL2.
NaBL2 does not have the power/expressiveness to allow implementing generics.
Furthermore, NaBL2 has limited static error reporting and limited ways to provide debugging information in case of runtime errors, which makes development in NaBL2 tedious.

\subsubsection{Compiler}
\label{subsubsec:problem_analysis__problems__compiler}

The \ac{PIE} \ac{DSL} had two compilers.
The first one compiled to Kotlin.
Kotlin is a language that gets compiled to bytecode for the JVM, which means that it can interoperate with Java and thus with the \ac{PIE} framework.
Oracle didn't want to introduce Kotlin to their environment and asked us to compile directly to Java.

The second compiler compiled to Java using string interpolation.
Using string interpolation has several problems.
First of all, it is error-prone because the Stratego compiler cannot check that the string would form valid Java code, which means that typos don't get caught until Java compile time.
The second issue is that it is inefficient when you want to do post-processing on the generated Java code.
The generated Java string needs to be parsed before it can be used.
Instead of compiling to a string and then parsing that string to a Java AST, it is more efficient to compile directly to a Java AST.
Compiling to an AST also somewhat solves the first issue.
The Stratego compiler can catch typos in constructor names, but it does not check that the generated AST is a valid Java AST.

All this leads to two requirements for the compiler: it needs to compile to something that is already used in the Spoofax ecosystem (i.e. Java or bytecode) and it needs to compile to an AST.

